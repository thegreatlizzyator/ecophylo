% Options for packages loaded elsewhere
\PassOptionsToPackage{unicode}{hyperref}
\PassOptionsToPackage{hyphens}{url}
%
\documentclass[
]{article}
\usepackage{lmodern}
\usepackage{amsmath}
\usepackage{ifxetex,ifluatex}
\ifnum 0\ifxetex 1\fi\ifluatex 1\fi=0 % if pdftex
  \usepackage[T1]{fontenc}
  \usepackage[utf8]{inputenc}
  \usepackage{textcomp} % provide euro and other symbols
  \usepackage{amssymb}
\else % if luatex or xetex
  \usepackage{unicode-math}
  \defaultfontfeatures{Scale=MatchLowercase}
  \defaultfontfeatures[\rmfamily]{Ligatures=TeX,Scale=1}
\fi
% Use upquote if available, for straight quotes in verbatim environments
\IfFileExists{upquote.sty}{\usepackage{upquote}}{}
\IfFileExists{microtype.sty}{% use microtype if available
  \usepackage[]{microtype}
  \UseMicrotypeSet[protrusion]{basicmath} % disable protrusion for tt fonts
}{}
\makeatletter
\@ifundefined{KOMAClassName}{% if non-KOMA class
  \IfFileExists{parskip.sty}{%
    \usepackage{parskip}
  }{% else
    \setlength{\parindent}{0pt}
    \setlength{\parskip}{6pt plus 2pt minus 1pt}}
}{% if KOMA class
  \KOMAoptions{parskip=half}}
\makeatother
\usepackage{xcolor}
\IfFileExists{xurl.sty}{\usepackage{xurl}}{} % add URL line breaks if available
\IfFileExists{bookmark.sty}{\usepackage{bookmark}}{\usepackage{hyperref}}
\hypersetup{
  pdftitle={Vignette ecophylo},
  pdfauthor={Team bouclé},
  hidelinks,
  pdfcreator={LaTeX via pandoc}}
\urlstyle{same} % disable monospaced font for URLs
\usepackage[margin=1in]{geometry}
\usepackage{color}
\usepackage{fancyvrb}
\newcommand{\VerbBar}{|}
\newcommand{\VERB}{\Verb[commandchars=\\\{\}]}
\DefineVerbatimEnvironment{Highlighting}{Verbatim}{commandchars=\\\{\}}
% Add ',fontsize=\small' for more characters per line
\usepackage{framed}
\definecolor{shadecolor}{RGB}{248,248,248}
\newenvironment{Shaded}{\begin{snugshade}}{\end{snugshade}}
\newcommand{\AlertTok}[1]{\textcolor[rgb]{0.94,0.16,0.16}{#1}}
\newcommand{\AnnotationTok}[1]{\textcolor[rgb]{0.56,0.35,0.01}{\textbf{\textit{#1}}}}
\newcommand{\AttributeTok}[1]{\textcolor[rgb]{0.77,0.63,0.00}{#1}}
\newcommand{\BaseNTok}[1]{\textcolor[rgb]{0.00,0.00,0.81}{#1}}
\newcommand{\BuiltInTok}[1]{#1}
\newcommand{\CharTok}[1]{\textcolor[rgb]{0.31,0.60,0.02}{#1}}
\newcommand{\CommentTok}[1]{\textcolor[rgb]{0.56,0.35,0.01}{\textit{#1}}}
\newcommand{\CommentVarTok}[1]{\textcolor[rgb]{0.56,0.35,0.01}{\textbf{\textit{#1}}}}
\newcommand{\ConstantTok}[1]{\textcolor[rgb]{0.00,0.00,0.00}{#1}}
\newcommand{\ControlFlowTok}[1]{\textcolor[rgb]{0.13,0.29,0.53}{\textbf{#1}}}
\newcommand{\DataTypeTok}[1]{\textcolor[rgb]{0.13,0.29,0.53}{#1}}
\newcommand{\DecValTok}[1]{\textcolor[rgb]{0.00,0.00,0.81}{#1}}
\newcommand{\DocumentationTok}[1]{\textcolor[rgb]{0.56,0.35,0.01}{\textbf{\textit{#1}}}}
\newcommand{\ErrorTok}[1]{\textcolor[rgb]{0.64,0.00,0.00}{\textbf{#1}}}
\newcommand{\ExtensionTok}[1]{#1}
\newcommand{\FloatTok}[1]{\textcolor[rgb]{0.00,0.00,0.81}{#1}}
\newcommand{\FunctionTok}[1]{\textcolor[rgb]{0.00,0.00,0.00}{#1}}
\newcommand{\ImportTok}[1]{#1}
\newcommand{\InformationTok}[1]{\textcolor[rgb]{0.56,0.35,0.01}{\textbf{\textit{#1}}}}
\newcommand{\KeywordTok}[1]{\textcolor[rgb]{0.13,0.29,0.53}{\textbf{#1}}}
\newcommand{\NormalTok}[1]{#1}
\newcommand{\OperatorTok}[1]{\textcolor[rgb]{0.81,0.36,0.00}{\textbf{#1}}}
\newcommand{\OtherTok}[1]{\textcolor[rgb]{0.56,0.35,0.01}{#1}}
\newcommand{\PreprocessorTok}[1]{\textcolor[rgb]{0.56,0.35,0.01}{\textit{#1}}}
\newcommand{\RegionMarkerTok}[1]{#1}
\newcommand{\SpecialCharTok}[1]{\textcolor[rgb]{0.00,0.00,0.00}{#1}}
\newcommand{\SpecialStringTok}[1]{\textcolor[rgb]{0.31,0.60,0.02}{#1}}
\newcommand{\StringTok}[1]{\textcolor[rgb]{0.31,0.60,0.02}{#1}}
\newcommand{\VariableTok}[1]{\textcolor[rgb]{0.00,0.00,0.00}{#1}}
\newcommand{\VerbatimStringTok}[1]{\textcolor[rgb]{0.31,0.60,0.02}{#1}}
\newcommand{\WarningTok}[1]{\textcolor[rgb]{0.56,0.35,0.01}{\textbf{\textit{#1}}}}
\usepackage{graphicx}
\makeatletter
\def\maxwidth{\ifdim\Gin@nat@width>\linewidth\linewidth\else\Gin@nat@width\fi}
\def\maxheight{\ifdim\Gin@nat@height>\textheight\textheight\else\Gin@nat@height\fi}
\makeatother
% Scale images if necessary, so that they will not overflow the page
% margins by default, and it is still possible to overwrite the defaults
% using explicit options in \includegraphics[width, height, ...]{}
\setkeys{Gin}{width=\maxwidth,height=\maxheight,keepaspectratio}
% Set default figure placement to htbp
\makeatletter
\def\fps@figure{htbp}
\makeatother
\setlength{\emergencystretch}{3em} % prevent overfull lines
\providecommand{\tightlist}{%
  \setlength{\itemsep}{0pt}\setlength{\parskip}{0pt}}
\setcounter{secnumdepth}{-\maxdimen} % remove section numbering
\ifluatex
  \usepackage{selnolig}  % disable illegal ligatures
\fi

\title{Vignette ecophylo}
\author{Team bouclé}
\date{9 janvier 2021}

\begin{document}
\maketitle

\hypertarget{rules-for-working-in-rmarkdown-with-python.}{%
\subsection{Rules for working in Rmarkdown with
python.}\label{rules-for-working-in-rmarkdown-with-python.}}

\emph{we will need to check the setup chunk for location of python3
binary across different os (ubuntu work with this code)}

\hypertarget{python-chunks-calling-r-obj.}{%
\subsubsection{Python chunks calling R
obj.}\label{python-chunks-calling-r-obj.}}

R chunks works just fine as usual.

\begin{Shaded}
\begin{Highlighting}[]
\NormalTok{a }\OtherTok{\textless{}{-}} \DecValTok{42} \CommentTok{\# set a in R}
\end{Highlighting}
\end{Shaded}

This chunk only work when the markdown is knitted. See that we can call
R object when looking for them in `r' namespace.

\begin{Shaded}
\begin{Highlighting}[]
\BuiltInTok{print}\NormalTok{(r.a)}
\end{Highlighting}
\end{Shaded}

\begin{verbatim}
## 42.0
\end{verbatim}

\begin{Shaded}
\begin{Highlighting}[]
\NormalTok{r.a }\OperatorTok{+=} \DecValTok{1} \CommentTok{\# modify R variable in python}
\BuiltInTok{print}\NormalTok{(r.a)}
\end{Highlighting}
\end{Shaded}

\begin{verbatim}
## 43.0
\end{verbatim}

\begin{Shaded}
\begin{Highlighting}[]
\NormalTok{b }\OperatorTok{=} \DecValTok{666} \CommentTok{\# set b in python}
\end{Highlighting}
\end{Shaded}

\hypertarget{r-chunks-calling-python-obj.}{%
\subsubsection{R chunks calling Python
obj.}\label{r-chunks-calling-python-obj.}}

See that we can call python object when looking for them in `py' list.

\begin{Shaded}
\begin{Highlighting}[]
\FunctionTok{print}\NormalTok{(a) }\CommentTok{\# call a in R after modification in python}
\end{Highlighting}
\end{Shaded}

\begin{verbatim}
## [1] 43
\end{verbatim}

\begin{Shaded}
\begin{Highlighting}[]
\FunctionTok{print}\NormalTok{(py}\SpecialCharTok{$}\NormalTok{b) }\CommentTok{\# call b in R}
\end{Highlighting}
\end{Shaded}

\begin{verbatim}
## [1] 666
\end{verbatim}

\begin{Shaded}
\begin{Highlighting}[]
\CommentTok{\# Also possible to call directly python function in r}
\NormalTok{rand }\OtherTok{\textless{}{-}} \FunctionTok{import}\NormalTok{(}\StringTok{\textquotesingle{}numpy\textquotesingle{}}\NormalTok{)}\SpecialCharTok{$}\NormalTok{random }\CommentTok{\# import python module numpy and submod random}
\NormalTok{rand}\SpecialCharTok{$}\FunctionTok{randint}\NormalTok{(}\DecValTok{3}\NormalTok{) }\CommentTok{\# call of function randint}
\end{Highlighting}
\end{Shaded}

\begin{verbatim}
## [1] 2
\end{verbatim}

You can run python code in interactive mod in the console using the
following function : \emph{repl\_python(quiet = T)}. It set the console
in `python mode' so you can type in python commands. However you can't
send lines from a chunk or a script.

To exit the consol in python, just type \emph{exit} and tadaa the consol
is back to R.

\hypertarget{using-python-modules-in-r.}{%
\subsubsection{Using Python modules in
R.}\label{using-python-modules-in-r.}}

\begin{Shaded}
\begin{Highlighting}[]
\CommentTok{\# Be carefull for idiotproof of ecophylo and int class !}
\NormalTok{eco }\OtherTok{\textless{}{-}} \FunctionTok{import}\NormalTok{(}\StringTok{\textquotesingle{}ecophylo\textquotesingle{}}\NormalTok{)}
\NormalTok{eco}\SpecialCharTok{$}\FunctionTok{timeframes}\NormalTok{(}\AttributeTok{I=}\FunctionTok{as.integer}\NormalTok{(}\DecValTok{3}\NormalTok{), }\AttributeTok{T=}\DecValTok{2}\NormalTok{, }\AttributeTok{a=}\FloatTok{0.3}\NormalTok{)}
\end{Highlighting}
\end{Shaded}

\begin{verbatim}
## [[1]]
## [1] 0.565357
## 
## [[2]]
## [1] 1.226603
## 
## [[3]]
## [1] 2
\end{verbatim}

\begin{Shaded}
\begin{Highlighting}[]
\CommentTok{\# make dataframes with pandas is same as R}
\NormalTok{pd }\OtherTok{\textless{}{-}} \FunctionTok{import}\NormalTok{(}\StringTok{\textquotesingle{}pandas\textquotesingle{}}\NormalTok{)}
\NormalTok{d }\OtherTok{\textless{}{-}} \FunctionTok{list}\NormalTok{(}\AttributeTok{col1 =} \FunctionTok{c}\NormalTok{(}\DecValTok{1}\NormalTok{,}\DecValTok{2}\NormalTok{,}\DecValTok{3}\NormalTok{), }\AttributeTok{col2 =} \FunctionTok{c}\NormalTok{(}\DecValTok{4}\NormalTok{,}\DecValTok{5}\NormalTok{,}\DecValTok{6}\NormalTok{))}
\NormalTok{pd}\SpecialCharTok{$}\FunctionTok{DataFrame}\NormalTok{(d)}
\end{Highlighting}
\end{Shaded}

\begin{verbatim}
##   col1 col2
## 1    1    4
## 2    2    5
## 3    3    6
\end{verbatim}

\newpage

\hypertarget{serious-part}{%
\section{Serious Part}\label{serious-part}}

\hypertarget{introduction}{%
\subsection{Introduction}\label{introduction}}

We introduce the Python package ecophylo dedicated to the
coalescent-based simulation of neutral eco-evolutionary dynamics.
Species assemblages and their shared ancestry can be simulated by
jointly taking into account the influence of past demographic
fluctuations and extinctions along with how divergent genotypes have
introduced new species over time through speciation.

The model rests on two main components: (i) a demographic component
driving stochastic changes in population sizes, structure and
extinctions due to habitat availability possibly linked migration
events, (ii) a mutation and protracted speciation component representing
how divergent genotypes emerge and define new species over time.

Here we show how to simulate communities and their phylogenetic
relationships for a wide range of past demographic scenarios having
affected whole assemblages of species. We also provide ways to produce
many simulated datasets from prior distributions so as to allow
Approximate Bayesian Computation (ABC) methods, to retrieve likely
parameter values from the comparison of these simulations to observed
diversity patterns.

\hypertarget{installation}{%
\subsection{Installation}\label{installation}}

The packages can be installed from the public repository Pypi with the
\texttt{pip} tool.

You can also download the \emph{tar.gz} file from the repository and
install it with pip. You need to use python3 and install the
dependencies before. Note that it is a development stage.

\textbf{impossible at this moment as the repository is private}

\begin{Shaded}
\begin{Highlighting}[]
\ExtensionTok{python3}\NormalTok{ {-}m pip install ecophylo}
\CommentTok{\# NOT RUN}
\CommentTok{\# wget {-}L https://github.com/thegreatlizzyator/ecophylo/blob/master/dist/ecophylo{-}0.0.5.tar.gz}
\CommentTok{\# python3 {-}m pip install ecophylo{-}0.0.5.tar.gz}
\end{Highlighting}
\end{Shaded}

\hypertarget{installation-on-windows-the-painfull-way}{%
\subsubsection{Installation on Windows (the painfull
way)}\label{installation-on-windows-the-painfull-way}}

\begin{quote}
For François : First create a new project in rstudio with
\textbf{version control} from the
\href{https://github.com/thegreatlizzyator/ecophylo}{github}.
\end{quote}

You will need to install \texttt{\{reticulate\}} package and python
dependencies.

\begin{Shaded}
\begin{Highlighting}[]
\CommentTok{\# NOT RUN}
\FunctionTok{install.packages}\NormalTok{(}\StringTok{\textquotesingle{}reticulate\textquotesingle{}}\NormalTok{) }\CommentTok{\# answer yes}
\FunctionTok{library}\NormalTok{(reticulate)}
\CommentTok{\# python dependencies}
\FunctionTok{conda\_install}\NormalTok{(}\StringTok{\textquotesingle{}r{-}reticulate\textquotesingle{}}\NormalTok{, }\FunctionTok{c}\NormalTok{(}\StringTok{\textquotesingle{}msprime\textquotesingle{}}\NormalTok{,}\StringTok{\textquotesingle{}ete3\textquotesingle{}}\NormalTok{,}\StringTok{\textquotesingle{}pandas\textquotesingle{}}\NormalTok{))}
\end{Highlighting}
\end{Shaded}

\hypertarget{single-simulation}{%
\subsection{Single simulation}\label{single-simulation}}

\begin{Shaded}
\begin{Highlighting}[]
\CommentTok{\# First we need to import the python module }
\NormalTok{eco }\OtherTok{\textless{}{-}} \FunctionTok{import}\NormalTok{(}\StringTok{\textquotesingle{}ecophylo\textquotesingle{}}\NormalTok{)}
\CommentTok{\# Be carefull for idiotproof of ecophylo and int class ! we need to }
\NormalTok{eco}\SpecialCharTok{$}\FunctionTok{timeframes}\NormalTok{(}\AttributeTok{I=}\FunctionTok{as.integer}\NormalTok{(}\DecValTok{3}\NormalTok{), }\AttributeTok{T=}\DecValTok{2}\NormalTok{, }\AttributeTok{a=}\FloatTok{0.3}\NormalTok{)}
\end{Highlighting}
\end{Shaded}

\begin{verbatim}
## [[1]]
## [1] 0.565357
## 
## [[2]]
## [1] 1.226603
## 
## [[3]]
## [1] 2
\end{verbatim}

\begin{Shaded}
\begin{Highlighting}[]
\NormalTok{sim1 }\OtherTok{\textless{}{-}}\NormalTok{ eco}\SpecialCharTok{$}\FunctionTok{simulate}\NormalTok{(}\FunctionTok{as.integer}\NormalTok{(}\DecValTok{10}\NormalTok{), }\FunctionTok{as.integer}\NormalTok{(}\FloatTok{1e5}\NormalTok{), }\FunctionTok{as.numeric}\NormalTok{(}\FloatTok{0.03}\NormalTok{), }\AttributeTok{seed =} \FunctionTok{as.integer}\NormalTok{(}\DecValTok{42}\NormalTok{))}
\NormalTok{sim1}
\end{Highlighting}
\end{Shaded}

\begin{verbatim}
## 
##       /-1
##    /-|
##   |  |   /-8
##   |   \-|
## --|      \-0
##   |
##   |   /-7
##   |  |
##    \-|      /-5
##      |   /-|
##       \-|   \-3
##         |
##          \-6
\end{verbatim}

\hypertarget{multiple-simulations}{%
\subsection{Multiple simulations}\label{multiple-simulations}}

\begin{Shaded}
\begin{Highlighting}[]
\CommentTok{\# NOT RUN}
\NormalTok{truc }\OtherTok{\textless{}{-}}\NormalTok{ eco}\SpecialCharTok{$}\FunctionTok{dosimuls}\NormalTok{(}\AttributeTok{nsim =} \DecValTok{1}\NormalTok{, }\AttributeTok{sample\_size =} \DecValTok{100}\NormalTok{, }\AttributeTok{comprior =} \FunctionTok{list}\NormalTok{(}\DecValTok{1000}\NormalTok{,}\FloatTok{10e9}\NormalTok{),}
             \AttributeTok{muprior =} \FunctionTok{list}\NormalTok{(}\FloatTok{1e{-}3}\NormalTok{) , }\AttributeTok{verbose =} \ConstantTok{TRUE}\NormalTok{, }\AttributeTok{seed =} \DecValTok{42}\NormalTok{)}
\end{Highlighting}
\end{Shaded}

\begin{Shaded}
\begin{Highlighting}[]
\CommentTok{\# NOT RUN}
\NormalTok{eco.dosimuls(nsim }\OperatorTok{=} \DecValTok{5}\NormalTok{, sample\_size }\OperatorTok{=} \DecValTok{100}\NormalTok{, comprior }\OperatorTok{=}\NormalTok{ [}\DecValTok{1000}\NormalTok{,}\FloatTok{10e9}\NormalTok{], }
\NormalTok{             muprior }\OperatorTok{=}\NormalTok{ [}\FloatTok{1e{-}3}\NormalTok{] , verbose }\OperatorTok{=} \VariableTok{True}\NormalTok{, seed }\OperatorTok{=} \DecValTok{42}\NormalTok{)}
\end{Highlighting}
\end{Shaded}


\end{document}
